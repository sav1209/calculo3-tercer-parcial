

Queremos calcular la integral:

\[
\int_{W} \sqrt{x^2 + y^2 + z^2} \cdot e^{-(x^2 + y^2 + z^2)} \, dV
\]

donde \( W \) es el sólido delimitado por las esferas \( x^2 + y^2 + z^2 = a^2 \) y \( x^2 + y^2 + z^2 = b^2 \), con \( 0 < b < a \).

Donde \( x^2 + y^2 + z^2 = a^2 \) y \( x^2 + y^2 + z^2 = b^2 \) son esferas concentricas, donde \( b \leq a \)

\subsection*{Representación en coordenadas esféricas}

En coordenadas esféricas, tenemos las relaciones:

\[
x^2 + y^2 + z^2 = r^2, \quad dV = r^2 \sin\theta \, dr \, d\theta \, d\phi
\]

Donde sustituimos las relaciones entre \(x\), \(y\), \(z\) a las coordenadas esféricas \(r\), \(\theta\), \(\phi\), y son:

\[
\begin{aligned}
x &= r \sin\theta \cos\phi, \\
y &= r \sin\theta \sin\phi, \\
z &= r \cos\theta,
\end{aligned}
\]

donde:
\begin{itemize}
    \item \(r\) es la distancia radial desde el origen.
    \item \(\theta \in [0, \pi]\) es el ángulo polar medido desde el eje \(z\).
    \item \(\phi \in [0, 2\pi]\) es el ángulo azimutal medido desde el eje \(x\) en el plano \(xy\).
\end{itemize}

\subsection*{Transformación de la Integral de Cartesiana a Coordenadas Esféricas}

Para transformar la integral:

\[
I =\int_{W} \sqrt{x^2 + y^2 + z^2} \cdot e^{-(x^2 + y^2 + z^2)} \, dV
\]

hacia la forma en coordenadas esféricas.

Por lo tanto, el término \(\sqrt{x^2 + y^2 + z^2}\) en la integral se convierte en:

\[
\sqrt{x^2 + y^2 + z^2} = r
\]

\subsection*{El diferencial de volumen \(dV\)}

En coordenadas esféricas, el diferencial de volumen \(dV\) se expresa como:

\[
dV = r^2 \sin\theta \, dr \, d\theta \, d\phi
\]

\subsection*{Reemplazo en la integral original}

Reemplazamos \(\sqrt{(x^2 + y^2 + z^2} = r\), y reemplazamos \(dV = r^2 \sin\theta \, dr \, d\theta \, d\phi\) en la integral original:
   \[
   I = \int_{0}^{2\pi} \int_{0}^{\pi} \int_{b}^{a} r \cdot e^{-r^2} \cdot r^2 \sin\theta \, dr \, d\theta \, d\phi
   \]
\subsection*{Límites de integración de la región \(W\)}

Finalmente, los límites de integración en \(r\) van de \(b\) a \(a\), ya que el sólido está limitado por las esferas \(x^2 + y^2 + z^2 = a^2\) y \(x^2 + y^2 + z^2 = b^2\).

Entonces, la integral completa en coordenadas esféricas es:

   \[
   \int_{W} \sqrt{x^2 + y^2 + z^2} \cdot e^{-(x^2 + y^2 + z^2)} \, dV = \int_{0}^{2\pi} \int_{0}^{\pi} \int_{b}^{a} r \cdot e^{-r^2} \cdot r^2 \sin\theta \, dr \, d\theta \, d\phi
   \]

Reemplazamos en la integral original:

\[
\int_W \sqrt{x^2 + y^2 + z^2} \cdot e^{-(x^2 + y^2 + z^2)} \, dV = \int_{r=b}^{a} \int_{\theta=0}^{\pi} \int_{\phi=0}^{2\pi} r \cdot e^{-r^2} \cdot r^2 \sin\theta \, d\phi \, d\theta \, dr
\]

Simplificando:

\[
I = \int_{0}^{2\pi} \int_{0}^{\pi} \int_{b}^{a} r^3 \cdot e^{-r^2} \sin\theta \, dr \, d\theta \, d\phi
\]

\subsection*{Separación de variables}

Dado que los límites de integración son independientes, podemos separar la integral en tres factores:

\[
\int_W (x^2 + y^2 + z^2) e^{-(x^2 + y^2 + z^2)} \, dV = \left( \int_{\phi=0}^{2\pi} d\phi \right) \left( \int_{\theta=0}^{\pi} \sin\theta \, d\theta \right) \left( \int_{r=b}^{a} r^3 e^{-r^2} \, dr \right)
\]

\subsection*{Evaluando las integrales angulares}

1. Integral en \( \phi \):

\[
\int_{\phi=0}^{2\pi} d\phi = 2\pi
\]

2. Integral en \( \theta \):

\[
\int_{\theta=0}^{\pi} \sin\theta \, d\theta = \left[ -\cos\theta \right]_0^{\pi} = -\cos\pi + \cos0 = 2
\]

\subsection*{Evaluación de la integral radial}

Sea \( u = r^2 \), entonces \( du = 2r \, dr \) y \( dr = \frac{du}{2r} \):
\[
\int_{b}^{a} r^3 e^{-r^2} \, dr = \int_{b^2}^{a^2} \frac{u}{2} e^{-u} \, du = \frac{1}{2} \int_{b^2}^{a^2} u e^{-u} \, du
\]

Usamos integración por partes donde \( v = u \) y \( dw = e^{-u} du \):
\[
\int u e^{-u} \, du = -u e^{-u} + \int e^{-u} \, du = -u e^{-u} - e^{-u}
\]

Evaluamos desde \( b^2 \) hasta \( a^2 \):
\[
\left[ -u e^{-u} - e^{-u} \right]_{b^2}^{a^2} = \left[ -(a^2 e^{-a^2} + e^{-a^2}) \right] - \left[ -(b^2 e^{-b^2} + e^{-b^2}) \right]
\]

\[
= -(a^2 + 1)e^{-a^2} + (b^2 + 1)e^{-b^2}
\]

\subsection*{Sustituyendo en las integrales}

Sustituyendo las integrales angulares y radiales en la expresión original:

\[
\text{Resultado} = 2\pi \cdot 2 \cdot \frac{1}{2} \left[ -(a^2 + 1)e^{-a^2} + (b^2 + 1)e^{-b^2} \right]
\]

Simplificando:

\[
\text{Resultado} = 2\pi \left[ (b^2 + 1)e^{-b^2} - (a^2 + 1)e^{-a^2} \right]
\]
