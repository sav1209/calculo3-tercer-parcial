Definimos la región triangular. La región está limitada por tres líneas conectadas entre los vértices del triángulo:

- Para \((0,0)\) y \((1,0)\):
  \[
  y = 0
  \]

- Para \((0,0)\) y \(\left(\frac{1}{2},\frac{1}{2}\right)\):
  \[
  y = x
  \]

- Para \((1,0)\) y \(\left(\frac{1}{2},\frac{1}{2}\right)\):\\
  Pendiente: \(\frac{\Delta y}{\Delta x} = \frac{\frac{1}{2}-0}{\frac{1}{2}-1} = -1\),
  \[
  y = 1 - x
  \]

De aquí definimos los límites de integración:
\[
0 \leq x \leq 1, \quad 0 \leq y \leq 1-x
\]

Hasta ahora la integral queda como:
\[
\int_0^1 \int_0^{1-x} \cos\left(\pi \frac{x-y}{x+y}\right) \, dy \, dx
\]

Definimos las nuevas variables:
\[
u = x - y, \quad v = x + y
\]

Calculamos el Jacobiano para ajustar las diferenciales. Resolvemos el sistema:
\[
u = x - y, \quad v = x + y
\]
\[
x = u + y
\]

Sustituyendo en \(v\):
\[
v = u + y + y = u + 2y \quad \rightarrow \quad y = \frac{v - u}{2}
\]

Sustituyendo en \(x = u + y\):
\[
x = u + \frac{v - u}{2} = \frac{v + u}{2}
\]

Para el Jacobiano:
\[
dx \, dy = 
\begin{vmatrix}
\frac{\partial(x)}{\partial(u)} & \frac{\partial(x)}{\partial(v)}\\
\frac{\partial(y)}{\partial(u)} & \frac{\partial(y)}{\partial(v)}
\end{vmatrix}
\, dudv
\]

\[
J =
\begin{vmatrix}
\frac{1}{2} & \frac{1}{2}\\
-\frac{1}{2} & \frac{1}{2}
\end{vmatrix}
= \frac{1}{2}
\]

Transformamos los límites de la integral. Usamos los vértices originales:
- Para \((0,0)\):
  \[
  u = x - y = 0, \quad v = x + y = 0 \quad \rightarrow \quad (u, v) = (0, 0)
  \]

- Para \((1,0)\):
  \[
  u = x - y = 1, \quad v = x + y = 1 \quad \rightarrow \quad (u, v) = (1, 1)
  \]

- Para \(\left(\frac{1}{2}, \frac{1}{2}\right)\):
  \[
  u = x - y = 0, \quad v = x + y = 1 \quad \rightarrow \quad (u, v) = (0, 1)
  \]

Por lo tanto:
\[
0 \leq v \leq 1, \quad 0 \leq u \leq v
\]

Con estos cambios, la integral queda como:
\[
\int_0^1 \int_0^{v} \cos\left(\pi \frac{u}{v}\right) \frac{1}{2} \, du \, dv
\]
\[
= \frac{1}{2} \int_0^1 \int_0^{v} \cos\left(\pi \frac{u}{v}\right) \, du \, dv
\]

Ahora resolvemos la integral del tipo:
\[
\int \cos\left(\pi \frac{x}{y}\right) dx
\]

Por cambio de variable:
\[
u = \frac{\pi x}{y}, \quad du = \frac{\pi}{y} dx
\]

La solución es:
\[
\frac{y}{\pi} \int \cos(u) \, du = \frac{y}{\pi} \sin\left(\frac{\pi x}{y}\right) + C
\]

Aplicando a nuestra integral definida:
\[
\frac{1}{2} \int_0^1 \frac{v}{\pi} \sin\left(\frac{\pi u}{v}\right) \bigg|_0^v dv
\]
\[
= \frac{1}{2} \int_0^1 \frac{v}{\pi} \left(\sin(\pi) - \sin(0)\right) dv
\]
\[
= \frac{1}{2} \int_0^1 0 \, dv = 0
\]

Dado que el coseno es una función oscilatoria simétrica, los valores positivos y negativos se cancelan entre sí. Por lo tanto, el resultado es:

\[
\therefore \int_0^1 \int_0^{1-x} \cos\left(\pi \frac{x-y}{x+y}\right) \, dy \, dx = 0
\]
