Para encontrar el area dentro de la curva \( r = 1 + \sin\theta \), es necesario describir la función en coordenadas polares.

Es decir, el area que describe la distancía \( r \) que está acotada por la función \( r = 1 + \sin\theta \) ; donde \(\theta\) varía libremente de \( 0 \) a \( 2\pi \)

\[
A = \int \int_R \, dA, \quad dA = r \, dr \, d\theta.
\]

Los límites de integración se vuelven: 
- \( \theta \) from \( 0 \) to \( 2\pi \),
- \( r \) from \( 0 \) to \( 1 + \sin\theta \).

Reemplazando los limites de integración de la región \( R \):
\[
A = \int_{0}^{2\pi} \int_{0}^{1+\sin\theta} r \, dr \, d\theta.
\]

Resolvemos la integral interior con respecto a r (\( dr \)):
\[
\int_{0}^{1+\sin\theta} r \, dr = \left[ \frac{r^2}{2} \right]_0^{1+\sin\theta} = \frac{(1+\sin\theta)^2}{2}.
\]

Sustituimos en la integral exterior con respecto a  (\( \theta \))
\[
A = \int_{0}^{2\pi} \frac{(1 + \sin\theta)^2}{2} \, d\theta.
\]

Se expande el binomio al cuadrado \( (1 + \sin\theta)^2 \):
\[
(1 + \sin\theta)^2 = 1 + 2\sin\theta + \sin^2\theta.
\]

Se extrae la constante \( \frac{1}{2} \) de la integral:
\[
A = \frac{1}{2} \int_{0}^{2\pi} \left( 1 + 2\sin\theta + \sin^2\theta \right) \, d\theta.
\]

Se separa la integral:
\[
A = \frac{1}{2} \left[ \int_{0}^{2\pi} 1 \, d\theta + 2 \int_{0}^{2\pi} \sin\theta \, d\theta + \int_{0}^{2\pi} \sin^2\theta \, d\theta \right].
\]

Resolvemos cada una de las integrales, usando las formulas:

\[
\int_{0}^{2\pi} du = u, \int_{0}^{2\pi} \sin udu = -\cos u, \int_{0}^{2\pi} \sin^2 udu = \frac{u}{2} - \frac{1}{4} \sin 2u
\]

Se obtienen los siguientes valores:

\[
\left[ \theta \right]_0^{2\pi} = 2\pi, \left[ -\cos\theta \right]_0^{2\pi} = 0, \left[ \frac{1}{2} - \frac{1}{4} \sin 2\theta \right]_0^{2\pi} = \pi, 
\]

Se sustituyen los valores en las integrales:
\[
\int_{0}^{2\pi} 1 \, d\theta = 2\pi, \quad \int_{0}^{2\pi} \sin\theta \, d\theta = 0, \quad \int_{0}^{2\pi} \sin^2\theta \, d\theta = \pi.
\]

Sustituyendo los valores obtenidos:
\[
A = \frac{1}{2} \left[ 2\pi + 0 + \pi \right] = \frac{1}{2} \cdot 3\pi = \frac{3\pi}{2}.
\]

Finalmente, el area es:
\[
\boxed{\frac{3\pi}{2}}
\]
