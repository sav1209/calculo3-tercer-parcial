\section*{Identificar la Transformación para la Parte (a)}

Para la parte (a), observamos el término \(3x + 2y\) junto con \(\sin(x-y)\). Una sustitución útil aquí es establecer \(u = x - y\) y \(v = 3x + 2y\). Esta transformación se basa en la estructura de los términos para simplificar el integrando.

\section*{Calcular el Jacobiano para la Parte (a)}

Para encontrar el Jacobiano de la transformación \(u = x - y\) y \(v = 3x + 2y\), planteamos las ecuaciones:

\[
x = \frac{1}{5}(2u + v) \quad \text{y} \quad y = \frac{1}{5}(3x - v).
\]

Luego, calculamos el determinante de la matriz Jacobiana:

\[
J = 
\begin{vmatrix}
\frac{\partial x}{\partial u} & \frac{\partial x}{\partial v} \\\\
\frac{\partial y}{\partial u} & \frac{\partial y}{\partial v}
\end{vmatrix}
\]

Esto resulta en:

\[
J = \begin{vmatrix}
\frac{2}{5} & \frac{1}{5} \\\\
-\frac{3}{5} & \frac{1}{5}
\end{vmatrix}
= \frac{1}{5}.
\]

Por lo tanto, el Jacobiano es \(\frac{1}{5}\).

\section*{Identificar la Transformación para la Parte (b)}

Para la parte (b), observamos el término exponencial \(e^{-4x+7y}\) y el término trigonométrico \(\cos(7x - 2y)\). Aquí, establecemos \(u = -4x + 7y\) y \(v = 7x - 2y\), con el objetivo de simplificar el integrando.

\section*{Calcular el Jacobiano para la Parte (b)}

Para calcular el Jacobiano, expresamos \(x\) e \(y\) en términos de \(u\) y \(v\). Las ecuaciones son:

\[
x = \frac{v + 2u}{30}, \quad y = \frac{7v + 4u}{30}.
\]

El Jacobiano es:

\[
J = 
\begin{vmatrix}
\frac{\partial x}{\partial u} & \frac{\partial x}{\partial v} \\\\
\frac{\partial y}{\partial u} & \frac{\partial y}{\partial v}
\end{vmatrix}.
\]

Esto resulta en:

\[
J = 
\begin{vmatrix}
\frac{2}{30} & \frac{1}{30} \\\\
\frac{4}{30} & \frac{7}{30}
\end{vmatrix}
= \frac{10}{900} = \frac{1}{90}.
\]

Por lo tanto, el Jacobiano es \(\frac{1}{90}\).
