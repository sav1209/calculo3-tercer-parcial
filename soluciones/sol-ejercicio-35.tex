La transformación \( T(u, v) = (u^2 - v^2, 2uv) \) transforma el rectángulo \( 1 \leq u \leq 2 \), \( 1 \leq v \leq 3 \) del plano \( uv \) en una región \( R \) del plano \( xy \).

\begin{enumerate}
    \item[(a)] Mostrar que \( T \) es uno a uno.
    \item[(b)] Encontrar el área de \( R \) utilizando la fórmula de cambio de variables.
\end{enumerate}

\section*{Demostración de que la transformación es uno a uno}

Para demostrar que \( T \) es uno a uno, asumimos que existen dos puntos \((u_1, v_1)\) y \((u_2, v_2)\) tales que:
\[
T(u_1, v_1) = T(u_2, v_2).
\]
Esto implica:
\[
u_1^2 - v_1^2 = u_2^2 - v_2^2,
\]
\[
2u_1v_1 = 2u_2v_2.
\]
De la segunda ecuación, tenemos que \( u_1v_1 = u_2v_2 \). Al sustituir esto en la primera ecuación y resolver para \( u_1, v_1 \) en términos de \( u_2, v_2 \), concluimos que \( u_1 = u_2 \) y \( v_1 = v_2 \), demostrando que \( T \) es uno a uno.

\section*{Cálculo del Jacobiano}

El determinante del Jacobiano de \( T(u, v) \) se calcula como:
\[
J = \begin{vmatrix}
\frac{\partial x}{\partial u} & \frac{\partial x}{\partial v} \\
\frac{\partial y}{\partial u} & \frac{\partial y}{\partial v}
\end{vmatrix},
\]
donde \( x = u^2 - v^2 \) y \( y = 2uv \). Por lo tanto:
\[
J = \begin{vmatrix}
2u & -2v \\
2v & 2u
\end{vmatrix} = (2u)(2u) - (-2v)(2v) = 4u^2 + 4v^2.
\]

\section*{Configuración de la integral para el área}

El área de la región \( R \) está dada por la integral doble:
\[
\text{Área de } R = \iint_R \frac{1}{|J|} \, dA,
\]
donde \( |J| = 4u^2 + 4v^2 \). En términos de las coordenadas \( u, v \), la integral es:
\[
\text{Área de } R = \int_{u=1}^{2} \int_{v=1}^{3} \frac{1}{4u^2 + 4v^2} \, dv \, du.
\]

\section*{Resolución paso a paso}

\subsection*{Paso 1: Simplificar el integrando}

El integrando es:
\[
\frac{1}{4u^2 + 4v^2} = \frac{1}{4(u^2 + v^2)}.
\]

\subsection*{Paso 2: Configurar la integral interna}

La integral interna es:
\[
\int_{v=1}^{3} \frac{1}{u^2 + v^2} \, dv.
\]

\subsection*{Paso 3: Resolver la integral interna}

La integral de \(\frac{1}{u^2 + v^2}\) respecto a \( v \) es:
\[
\int \frac{1}{u^2 + v^2} \, dv = \frac{1}{u} \arctan\left(\frac{v}{u}\right).
\]
Evaluando de \( v=1 \) a \( v=3 \):
\[
\frac{1}{u} \left[\arctan\left(\frac{3}{u}\right) - \arctan\left(\frac{1}{u}\right)\right].
\]

\subsection*{Paso 4: Configurar la integral externa}

La integral externa se convierte en:
\[
\int_{u=1}^{2} \frac{1}{4} \cdot \frac{1}{u} \left[\arctan\left(\frac{3}{u}\right) - \arctan\left(\frac{1}{u}\right)\right] \, du.
\]

\subsection*{Paso 5: Resolver la integral externa}

Dividimos la integral en dos términos:
\[
\int_{u=1}^{2} \frac{1}{u} \arctan\left(\frac{3}{u}\right) \, du - \int_{u=1}^{2} \frac{1}{u} \arctan\left(\frac{1}{u}\right) \, du.
\]

Utilizando herramientas computacionales, se obtiene que el valor aproximado de la integral es:
\[
\text{Área de } R \approx 0.0722.
\]

\section*{Concepto clave}

Una integral doble evalúa el "volumen" bajo una superficie descrita por una función \( f(x, y) \) sobre una región específica en el plano \( xy \). El proceso involucra:

\begin{enumerate}
    \item \textbf{Integral interna:} Se fija una variable (por ejemplo, \( x \)) y se integra \( f(x, y) \) respecto a \( y \) en los límites dados, calculando un "corte" del volumen para un valor específico de \( x \).
    \item \textbf{Integral externa:} Se integra el resultado de la integral interna respecto a \( x \) en sus límites, sumando todos los cortes para obtener el volumen total.
\end{enumerate}

En este problema, se utilizó la función racional \( f(u, v) = \frac{1}{4(u^2 + v^2)} \), lo que requirió usar propiedades de la función arctangente durante la integración.
