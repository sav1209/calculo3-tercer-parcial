Tenemos que $dxdy = dA$

Ahora bien, si $x^2 + y^2 \leq 4$
$$x = \sqrt{4 - y^2} = \sqrt{4}$$
$$x = \pm 2$$
$$0 \leq x \leq 2$$
$$y = \sqrt{4 - x^2} = \sqrt{4}$$
$$y = \pm 2$$
$$0 \leq y \leq 2$$

Aplicamos coordenadas polares:
$$x^2 + y^2 = r^2$$
$$x^2 + y^2 \leq 4$$
$$r^2 = 4$$
$$r = \sqrt{4}$$
$$r = 2$$

Las funciones de movimiento de $0$ a $2\pi$ son un disco, entonces:
$$dxdy = dA = rdrd\theta$$
$$\int_0^2\int_0^{2\pi} (r^2)^{\frac{3}{2}} \ rd\theta dr$$
$$= \int_0^2\int_0^{2\pi} r^3 \ rd\theta dr = \int_0^2\int_0^{2\pi} r^4 \ d\theta dr$$
$$= \int_0^2 r^4\int_0^{2\pi} d\theta dr = \int_0^2 r^4 (\theta)\big|_0^{2\pi} dr$$
$$= \int_0^2 r^4 \cdot (2\pi)dr$$
$$= (2\pi) \cdot \int_0^2 r^4 = (2\pi) \cdot \frac{r^5}{5}\Big|_0^2$$
$$= (2\pi) \cdot \frac{2^5}{5} = 2\pi \cdot \frac{32}{5} = \frac{64\pi}{5} $$ 
