Para resolver nuestra integral haremos un cambio de variables a coordenadas esfericas donde:
$$
\begin{aligned}
 x=\rho \operatorname{sen} \phi \cos \theta\\
 y=\phi \operatorname{sen} \phi \operatorname{sen} \theta\\ 
 z=\rho \cos \phi \\
 \rho^2=x^2+y^2+z^2 \\
 dxdydz=\rho^2 \operatorname{sen} \phi d \rho d \phi d \theta \\
\end{aligned}
$$
Al ser $B$ una bola unitaria, definimos nuestros limites de la siguiente forma
$$
\begin{aligned}
& 0 \leq \phi \leq 1 \\
& 0 \leq \theta \leq 2 \pi \\
& 0 \leq \phi \leq \pi
\end{aligned}
$$

Reescribiendo la integral:
$$
\begin{aligned}
& \iiint_B \frac{d x d y d z}{\sqrt{2+x^2+y^2+z^2}}=\int_0^{2 \pi} \int_0^\pi \int_0^1 \frac{\rho^2 \operatorname{sen} \phi}{\sqrt{2+\rho^2}} d \rho d \phi d \theta
\end{aligned}
$$

Solucionando
$$
\begin{aligned}
& \int_0^1 \frac{\rho^2 \operatorname{sen} \phi}{\sqrt{2+\phi^2}} d \rho \Rightarrow \operatorname{sen} \phi \int_0^1 \frac{\rho^2}{\sqrt{2+\rho^2}} d \rho.
\end{aligned}
$$
definimos a $u$ y $\rho$:
$$
\rho=\sqrt{2} \tan (u) \Rightarrow u=\arctan \left(\frac{\rho}{u}\right), \quad d\rho=\sqrt{2} \sec ^2(u) d u
$$

Realizamos la sustitución en nuestra integral
$$
 \int \frac{2^{3 / 2} \sec ^2(u) \tan ^2(u)}{\sqrt{2 \tan ^2(u)+2}} d u
 $$
 $$
 2\tan^2(u)+2=2\sec^2(u)
 $$
 $$
 \Rightarrow  \int \frac{2^{3 / 2} \sec ^2(u) \tan ^2(u)}{\sqrt{2 \tan ^2(u)+2}}
 =2 \int \sec (u) \tan ^2(u) d u
 $$
 $$
 \tan ^2(u)=\sec ^2(u)-1
 $$
 
 $$
 \Rightarrow 2 \int \sec ^3(u)-\sec (u) d u\\
$$
$$
 \Rightarrow 2\left[\int \sec ^3(u) d u-\int \sec (u) d u\right]
$$
Resolviendo
$$
 \int \sec ^3(u) d u= \frac{\sec (u) \tan (u)}{2}+\frac{1}{2} \int \sec (u) d u
 $$
Resolviendo
$$
 \int \sec (u) d u=\ln (\tan (u)+\sec (u))
 $$
Reemplazamos las soluciones
$$
 2 \int {\sec (u) \tan ^2(u)} d u \\
 $$
 $$
 = -2[\frac{\sec(u)\tan(u)}{2}+\frac{\ln(\tan (u)+\sec(u)}{2}-\ln(\tan (u)+\sec (u))]\\
 $$
 $$
 = \sec (u) \tan (u)-\ln (\tan ( u)+\sec (u))
$$
$$
\text{Recordemos que }\\
u=\arctan(\frac{\rho}{\sqrt{2}})\\
$$
Y reemplazamos en nuestro resultado

$$
\Rightarrow \sec \left(\arctan \left(\frac{\rho}{\sqrt{2}}\right)\right) \tan \left(\arctan \left(\frac{\rho}{\sqrt{2}}\right)\right)-\operatorname{Ln}\left[\tan \left(\arctan \left(\frac{\rho}{\sqrt{2}}\right)\right)\ -\sec \left(\arctan \left(\frac{\rho}{\sqrt{2}}\right)\right)\right]\\
$$

$$
\sec \left(\arctan \left(\frac{\rho}{\sqrt{2}}\right)\right)=\sqrt{\frac{\rho^2}{2}+1}\\
\text{ y }\\
\tan \left(\arctan \left(\frac{\rho}{\sqrt{2}}\right)\right) = \frac{\rho}{2} \\
$$
$$
\Rightarrow \sqrt{\frac{p^2}{2}+1} \left(\frac{\rho}{\sqrt{2}}\right)-\ln \left[\tan \left(\arctan \left(\frac{\rho}{\sqrt{2}}\right)\right) -\sec \left(\arctan \left(\frac{\rho}{\sqrt{2}}\right)\right)\right]
$$

$$
=\frac{\rho \sqrt{\rho^2+2}-2 \ln \left(1 \sqrt{\rho^2+2}+\rho \mid\right.}{2} \\
=  \frac{\rho \sqrt{\rho^2+2}}{2}-\operatorname{arcsenh}\left(\frac{\rho}{\sqrt{2}}\right) \\
$$
$$
\Rightarrow \int_0^1 \frac{\rho^2 \operatorname{sen} \phi}{\sqrt{2+\rho^2}} d \rho=\operatorname{sen} \phi\left[\frac{\rho \sqrt{\rho^2+2}}{2}-\left.\operatorname{arcsenh}\left(\frac{\rho}{\sqrt{2}}\right)\right|_0 ^1\right] \\
$$
$$
=  \operatorname{sen} \phi\left[\frac{\sqrt{3}}{2}-\operatorname{arcsenh}\left(\frac{1}{\sqrt{2}}\right)\right]
$$

Remplazando el resultado en la integral triple Inicial
$$
\int_0^{2 \pi} \int_0^\pi \operatorname{sen} \phi\left[\frac{\sqrt{3}}{2}-\operatorname{arcsen}\left(\frac{1}{\sqrt{2}}\right)\right] d \phi d \theta
$$

Ahora resolviendo
$$
\begin{aligned}
& \int_0^\pi \operatorname{sen} \phi\left[\frac{\sqrt{3}}{2}-\operatorname{arcsen}\left(\frac{1}{\sqrt{2}}\right)\right] d \phi=\left[\frac{\sqrt{3}}{2}-\operatorname{arcshh}\left(\frac{1}{\sqrt{2}}\right)\right] \int_0^\pi \operatorname{sen} \phi d \phi \\
&= {\left[\frac{\sqrt{3}}{2}-\operatorname{arcsenh}\left(\frac{1}{\sqrt{2}}\right)\left(-\cos (\pi)+\cos (0)=\sqrt{3}-2 \operatorname{arcsenh}\left(\frac{1}{\sqrt{2}}\right)\right.\right.} \\
\end{aligned}
$$

Por ultimo tenemos
$$
\begin{array}{l}
\int_0^{2\pi} \sqrt{3}-2 \operatorname{arcsenh}\left(\frac{1}{\sqrt{2}}\right) d \theta=\left.\theta\left(\sqrt{3}-\operatorname{arcsenh}\left(\frac{1}{\sqrt{2}}\right)\right)\right|_0 ^{2 \pi} \\
=2 \pi\left(\sqrt{3}-\operatorname{arcsenh}\left(\frac{1}{\sqrt{2}}\right)\right) \approx 2.69649
\end{array}
$$
Finalmente el resultado de
$$
\int_0^{2 \pi} \int_0^\pi \int_0^1 \frac{\rho^2 \operatorname{sen} \phi}{\sqrt{2+\rho^2}} d \rho d \phi d \theta  \approx 2.69649
$$
