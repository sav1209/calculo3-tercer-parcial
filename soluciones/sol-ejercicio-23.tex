Soluciōn
Para resolver nuestra integral haremos un cambio de variables a coordenadas esfericas dönde:
$$
\begin{array}{ll}
x=\phi \operatorname{sen} \phi \cos \theta & d x d y d z=p^2 \operatorname{sen} \phi d p d \phi d \theta \\
y=\phi \operatorname{sen} \phi \operatorname{sen} \theta & p^2=x^2+y^2+z^2 . \\
z=\phi \cos \phi & 1
\end{array}
$$

Al ser $B$ una bola unitaria, definimos nuestros limites de la siguiente forma
$$
\begin{aligned}
& 0 \leq \phi \leq 1 \\
& 0 \leq \theta \leq 2 \pi \\
& 0 \leq \Phi \leq \pi
\end{aligned}
$$

Reescribiendo la integral:
$$
\begin{aligned}
& \iiint_B \frac{d x d y d z}{\sqrt{2+x^2+y^2+z^2}}=\int_0^{2 \pi} \int_0^\pi \int_0^1 \frac{\phi^2 \operatorname{sen} \phi}{\sqrt{2+p^2}} d p d \Phi d \theta
\end{aligned}
$$

Solucionando
$$
\begin{aligned}
& \text { olucion ando } \\
& \int_0^1 \frac{p^2 \operatorname{sen} \phi}{\sqrt{2+\phi^2}} d \rho \Rightarrow \operatorname{sen} \phi \int_0^1 \frac{\phi^2}{\sqrt{2+\phi^2}} d \phi.
\end{aligned}
$$
definimos a
$$
Phi=\sqrt{2} \tan (u) \Rightarrow u=\arctan \left(\frac{phi}{u}\right), \quad d phi=\sqrt{2} \sec ^2(u) d u
$$

Realizamos la sustitución en nuestra integral
$$
\begin{aligned}
& \int \frac{2^{3 / 2} \sec ^2(u) \tan ^2(u)}{\sqrt{2 \tan ^2(u)+2}} d u \\
& 2 \tan ^2(u)+2=2 \sec ^2(u) \\
\Rightarrow & \int \frac{2^{3 / 2} \sec ^2(u) \tan ^2(u)}{\sqrt{2 \tan ^2(u)+2}}=2 \int \sec (u) \tan ^2(u) d u \\
& \tan ^2(u)=\sec ^2(u)-1 \\
\Rightarrow & 2 \int \sec ^3(u)-\sec (u) d u
\end{aligned}
$$

Aplicando linealidad
$$
\begin{aligned}
& 2\left[\int \sec ^3(u) d u-\int \sec (u) d u\right] \\
& \text { Resolviendo } \\
& \int \sec ^3(u) d u= \frac{\sec (u) \tan (u)}{2}+\frac{1}{2} \int \sec (u) d u \\
& \text { Resolviendo } \\
& \int \sec (u) d u=\ln (\tan (u)+\sec (u))
\end{aligned}
$$
$$
\begin{aligned}
& \text { Reemplazamos las soluciones } \\
& 2 \int \sec (u) \tan ^2(u) d u \\
& -2\left[\frac{\sec (u) \tan (u)}{2}+\frac{\ln (\tan (u)+\sec (u)}{2}-\ln (\tan (u)+\sec (u))]\right. \\
& =\sec (u) \tan (u)-\ln (\tan ( u)+\sec (u))
\end{aligned}
$$
