Solucion\\
Con ayuda de las condiciones dadas obtenemos los siguientes limites de integración
$$
\begin{gathered}
0 \leq x \leq 1 \\
(1-x) \leq y \leq \sqrt{1-x^2}
\end{gathered}
$$
y podemos establecer la integral doble dela siguiente forma
$$
\int_0^1 \int_{1-x}^{\sqrt{1-x^2}} \frac{1}{\left(x^2+y^2\right)^2} d y d x
$$

Para resolverlo hacemos cambio a coordenadas
$$
\begin{aligned}
& \int_0^1 \int_{1-x}^{\sqrt{1-x^2}} \frac{1}{\left(x^2+y^2\right)^2} d y d x\Rightarrow \int_0^{\pi / 2} \int_{\frac{1}{\cos \theta+\operatorname{sen} \theta}}^1{\frac{1}{\left(r^2\right)^2} r d r d \theta} \\
\text{Resolviendo}\\
& \int_{\frac{1}{\cos \theta+\operatorname{sen} \theta}}^1{\frac{1}
{\left(r^2\right)^2} r d r} \int_{\frac{1}{\cos \theta+\operatorname{sen} \theta}}^1 \frac{1}{r^3} d r=-\left.\frac{1}{2 x^2}\right|_{\frac{1}{\cos \theta+\operatorname{sen} \theta}}^1
\end{aligned}
$$
$$
\begin{aligned}
& =-\frac{1}{2(1)^2}+\frac{1}{2\left(\frac{1}{(\cos \theta+\operatorname{sen} \theta)^2}\right).} \\
& =\frac{(\cos \theta+\operatorname{sen} \theta)^2-1}{2}
\end{aligned}
$$

Remplazamos en nuestra integral
$$
\begin{aligned}
& \int_0^{\pi / 2} \int_{\frac{1}{\cos \theta+\operatorname{sen} \theta}}^1 \frac{1}{r^3} d r d \theta=\int_0^{\pi / 2} \frac{(\cos\theta+\operatorname{sen} \theta)^2-1}{2} d \theta \\
& (\cos \theta+\operatorname{sen} \theta)^2=1+2 \cos \theta \operatorname{sen} \theta \\
\Rightarrow & \int_0^{\pi / 2} \frac{(2\operatorname{cos}\theta \operatorname{sen} \theta+1)-1}{2} d \theta=\int_0^{\pi / 2} cos \theta \operatorname{sen} \theta d \theta
\end{aligned}
$$
$$
\begin{aligned}
\text{Sea}\\
&u=\operatorname{sen} \theta, d u=\cos \theta d \theta \\
\Rightarrow & \int u d u=\left.\frac{u^2}{2} \Rightarrow \frac{\operatorname{sen}^2(\theta)}{2}\right|_0 ^{\pi / 2} \\
& \frac{\operatorname{sen}^2(\pi / 2)}{2}=\frac{(1)^2}{2}=\frac{1}{2}
\end{aligned}
$$

Finalmente decimos que
$$
\int_0^{\pi / 2} \int_{\frac{1}{\cos \theta+\operatorname{sen} \theta}}^1 \frac{1}{r^3} d r d \theta=\frac{1}{2}
