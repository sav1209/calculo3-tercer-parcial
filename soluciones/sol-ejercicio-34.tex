

	Primero debemos definir los limites de integración:\\
	
	Para la coordenada radial $\rho$, la distancia del origen al punto tenemos que \( \rho = \sqrt{6} \). Lo qué significa que $\rho$ varia de 0 a $\sqrt{6}$. Por lo tanto, los limites para $\rho$ son: $0\leq \rho \leq \sqrt{6}$.\\
	
	Ahora bien, el angulo polar $\phi$, es el ángulo de la proyección del punto con respecto al eje z. En este caso, la región esta acotada por dos conos:
	
	\begin{center}
	\begin{enumerate}
	\item $ \phi = \frac{\pi}{4}$
	\item $ \phi = \arctan(2)$
	\end{enumerate}
	\end{center}
	
	Por lo tanto, los limites para $\phi$ son: $\frac{\pi}{4} \leq \phi \leq \arctan(2)$.\\
	
	Finalmente, $\theta$ el ángulo en el plano XY, esta limitado por: $\leq \theta \leq \frac{\pi}{2}$\\
	
	En coordenadas esféricas, el diferencial de volumen es:
	
	\[
	dV = \rho^2 \sin(\phi) \, d\rho \, d\phi \, d\theta.
	\]
	
	Habiendo definido los limites de integración para cada coordenada, podemos escribir la integral como:
	\[
	\iiint_{\text{R}} \frac{1}{\rho} \, dV = \int_{0}^{\frac{\pi}{2}} \int_{\frac{\pi}{4}}^{\arctan(2)} \int_{0}^{\sqrt{6}} \frac{1}{\rho} \rho^2 \sin(\phi) \, d\rho \, d\phi \, d\theta.
	\]
	
	Simplificando la función:
	\[
	\frac{1}{\rho} \rho^2 \sin(\phi) = \rho \sin(\phi),
	\]
	la integral la reescribimos como:
	\[
	\int_{0}^{\frac{\pi}{2}} \int_{\frac{\pi}{4}}^{\arctan(2)} \int_{0}^{\sqrt{6}} \rho \sin(\phi) \, d\rho \, d\phi \, d\theta.
	\]
	
	
	\subsection*{Integral respecto a \( \rho \)}
	\[
	\int_{0}^{\sqrt{6}} \rho \, d\rho = \left[ \frac{\rho^2}{2} \right]_{0}^{\sqrt{6}} = \frac{6}{2} = 3.
	\]
	
	Esto reduce la integral a:
	\[
	3 \int_{0}^{\frac{\pi}{2}} \int_{\frac{\pi}{4}}^{\arctan(2)} \sin(\phi) \, d\phi \, d\theta.
	\]
	
	\subsection*{Integral respecto a \( \phi \)}
	La integral respecto a \( \phi \) es:
	\[
	\int_{\frac{\pi}{4}}^{\arctan(2)} \sin(\phi) \, d\phi = \left[-\cos(\phi)\right]_{\frac{\pi}{4}}^{\arctan(2)}.
	\]
	
	Calculamos los valores:
	\[
	\cos\left(\frac{\pi}{4}\right) = \frac{\sqrt{2}}{2}, \quad \cos(\arctan(2)) = \frac{1}{\sqrt{5}}.
	\]
	
	Entonces:
	\[
	\int_{\frac{\pi}{4}}^{\arctan(2)} \sin(\phi) \, d\phi = -\frac{1}{\sqrt{5}} + \frac{\sqrt{2}}{2}.
	\]
	
	\subsection*{Integral respecto a \( \theta \)}
	La integral respecto a \( \theta \) es:
	\[
	\int_{0}^{\frac{\pi}{2}} 1 \, d\theta = \frac{\pi}{2}.
	\]
	
	\subsection*{Resultado:}
	El resultado de la integral es:
	\[
	\iiint_{\text{región}} \frac{1}{\rho} \, dV = 3 \cdot \frac{\pi}{2} \cdot \left(\frac{\sqrt{2}}{2} - \frac{1}{\sqrt{5}}\right).
	\]
	
	Así, el valor de la integral es:
	\[
	\boxed{\frac{3\pi}{2} \left(\frac{\sqrt{2}}{2} - \frac{1}{\sqrt{5}}\right)}.
	\]
	

