\section*{Cálculo del área de una lemniscata usando coordenadas polares}

La ecuación de la lemniscata es:
\[
(x^2 + y^2)^2 = 2a^2(x^2 - y^2)
\]

\subsection*{Paso 1: Cambiar a coordenadas polares}

En coordenadas polares, \(x = r\cos\theta\) y \(y = r\sin\theta\), además \(x^2 + y^2 = r^2\). Sustituimos en la ecuación de la lemniscata:
\[
(r^2)^2 = 2a^2(r^2\cos^2\theta - r^2\sin^2\theta)
\]

Esto se simplifica a:
\[
r^4 = 2a^2r^2(\cos^2\theta - \sin^2\theta)
\]

Factorizamos \(r^2\) (suponiendo \(r \neq 0\)):
\[
r^2 = 2a^2(\cos^2\theta - \sin^2\theta)
\]

Usamos la identidad \(\cos^2\theta - \sin^2\theta = \cos(2\theta)\):
\[
r^2 = 2a^2\cos(2\theta)
\]

Por lo tanto:
\[
r = \sqrt{2a^2\cos(2\theta)}
\]

\subsection*{Paso 2: Área en coordenadas polares}

La fórmula para el área en coordenadas polares es:
\[
A = \frac{1}{2} \int_{\theta_1}^{\theta_2} r^2 \, d\theta
\]

Sustituimos \(r^2 = 2a^2\cos(2\theta)\):
\[
A = \frac{1}{2} \int_{\theta_1}^{\theta_2} 2a^2\cos(2\theta) \, d\theta
\]

\[
A = a^2 \int_{\theta_1}^{\theta_2} \cos(2\theta) \, d\theta
\]

La lemniscata tiene simetría respecto al origen, por lo que podemos calcular el área de  ( \(-\frac{\pi}{4}\) a \(\frac{\pi}{4}\)) y luego multiplicar por 2:
\[
A_{\text{total}} = 2 \cdot a^2 \int_{-\pi/4}^{\pi/4} \cos(2\theta) \, d\theta
\]

\subsection*{Paso 3: Resolver la integral}

Sabemos que \(\int \cos(2\theta) \, d\theta = \frac{1}{2} \sin(2\theta)\):
\[
A_{\text{total}} = 2 \cdot a^2 \left[ \frac{1}{2} \sin(2\theta) \right]_{-\pi/4}^{\pi/4}
\]

Evaluamos los límites:
\[
\sin(2\theta) \text{ en } \theta = \frac{\pi}{4} \text{ es } \sin\left(\frac{\pi}{2}\right) = 1
\]
\[
\sin(2\theta) \text{ en } \theta = -\frac{\pi}{4} \text{ es } \sin\left(-\frac{\pi}{2}\right) = -1
\]

Por lo tanto:
\[
A_{\text{total}} = 2 \cdot a^2 \cdot \frac{1}{2} \left(1 - (-1)\right)
\]

\[
A_{\text{total}} = 2 \cdot a^2
\]

\subsection*{Respuesta final}

El área total delimitada por la lemniscata es:
\[
A = 2a^2
\]
