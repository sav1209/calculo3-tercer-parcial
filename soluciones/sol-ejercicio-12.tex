Queremos demostrar que:
$$\iint_D f(x, y) \ dx \ dy = \iint_{D^*} f(u, \psi(u, v))\begin{vmatrix}
    \frac{\partial \psi}{\partial v}
\end{vmatrix} dudv$$

Si tenemos que $x = u$ y $y = \psi(u, v)$
\\Entonces:
$$dx dy = \begin{vmatrix}
    \frac{\partial(x, y)}{\partial(u, v)}
\end{vmatrix} du dv$$

en donde $\frac{\partial(x, y)}{\partial(u, v)}$ es el Jacobiano de la transformación $T.$

Si calculamos el Jacobiano:
$$J_T = \begin{bmatrix}
    \frac{\partial x}{\partial u} & \frac{\partial x}{\partial v}\\
    \frac{\partial y}{\partial u} & \frac{\partial y}{\partial v}
\end{bmatrix} = \begin{bmatrix}
    1 & 0\\
    \frac{\partial \psi}{\partial u} & \frac{\partial \psi}{\partial v}
\end{bmatrix}$$

Así, el determinante de la matriz es:

$$\begin{vmatrix}
    J_T
\end{vmatrix} = \begin{vmatrix}
    \frac{\partial \psi}{\partial v}
\end{vmatrix}$$

Observación:
\\Como en $D^*$, $v$ está acotada por $g(u) \leq v \leq h(u)$ y a su vez, $u$ por $a \leq u \leq b.$

Entonces la transofrmación $T$ mapea la región en el plano $uv$ al dominio $D$ en el plano $xy,$ manteniendo así la correspondencia uno a uno, ya que teníamos que $\frac{\partial \psi}{\partial v} \neq 0.$

Así, dado que $\psi$ es de $C^{-1}, \frac{\partial \psi}{\partial v} \neq 0$ y $f$ es continua, el cambio de variables es completamente válido y la ecuación dada al inicio se cumple.

$$\iint_D f(x, y) \ dx \ dy = \iint_{D^*} f(u, \psi(u, v))\begin{vmatrix}
    \frac{\partial \psi}{\partial v}
\end{vmatrix} dudv \ \ \blacksquare$$
