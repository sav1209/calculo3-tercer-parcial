\documentclass{article}
\usepackage{amsmath, amssymb}
\usepackage[a4paper, left=2cm, right=2cm, top=2cm, bottom=2cm]{geometry}
\begin{document}
	
	\section*{Ejercicio 19}
	
	\[
	\int\int_R (x + y)^2 e^{x - y} \, dx \, dy,
	\]
	Donde \( R \) está delimitada por las líneas \( x + y = 1 \), \( x + y = 4 \), \( x - y = -1 \), y \( x - y = 1 \).
	
	\subsection*{Cambio de variable}
	Realizamos el cambio de variables:
	\[
	u = x + y, \quad v = x - y.
	\]
	Ahora bien, para asegurarnos de que el cambio es válido, debemos resolver el sistema $u=x+y$ y $v=x-y$ para $x$ y $y$:
	\[
	x = \frac{u + v}{2}, \quad y = \frac{u - v}{2}.
	\]
	
	Esto muestra que el cambio es invertible, es decir, cada par (u,v) corresponde a exactamente un par (x,y).\\
	
	El jacobiano del cambio es:
	\[
	\frac{\partial(x, y)}{\partial(u, v)} = 
	\begin{vmatrix}
		\frac{\partial x}{\partial u} & \frac{\partial x}{\partial v} \\
		\frac{\partial y}{\partial u} & \frac{\partial y}{\partial v}
	\end{vmatrix} =
	\begin{vmatrix}
		\frac{1}{2} & \frac{1}{2} \\
		\frac{1}{2} & -\frac{1}{2}
	\end{vmatrix} = -\frac{1}{2}.
	\]
	Por lo tanto, el valor absoluto del determinante es:
	\[
	\left| \frac{\partial(x, y)}{\partial(u, v)} \right| = \frac{1}{2}.
	\]
	
	Entonces, el diferencial se ajusta como: $dx \, dy$ = $\frac{1}{2}$ $\, du \, dv$
	
	
	\subsection*{Definimos la región en el plano \( (u, v) \)}
	Las ecuaciones que delimitan la región \( R \) en términos de \( u \) y \( v \) son:
	\begin{align*}
		x + y &= u \quad \Rightarrow \quad u \in [1, 4], \\
		x - y &= v \quad \Rightarrow \quad v \in [-1, 1].
	\end{align*}
	
Por lo tanto, en el plano \( (u, v) \), la región \( R \) es un rectángulo con:
	\[
	1 \leq u \leq 4, \quad -1 \leq v \leq 1.
	\]
	
	\subsection*{Reescribimos la integral aplicando el cambio de variable}
	Al realizar el cambio, la integral queda de la siguiente manera:
	\[
	 \int_{u=1}^{4} \int_{v=-1}^{1} u^2 e^v \cdot \frac{1}{2} \, dv \, du.
	\]
	
	\[
	 \frac{1}{2} \int_{u=1}^{4} \int_{v=-1}^{1} u^2 e^v \, dv \, du.
	\]
	
	\subsection*{Resolver la integral}
	\subsubsection*{Integral respecto a \( v \):}
	\[
	\int_{v=-1}^1 e^v \, dv = \left[ e^v \right]_{v=-1}^1 = e^1 - e^{-1}.
	\]
	
	\subsubsection*{Sustituir en la integral respecto a \( u \):}
	\[
	 \frac{1}{2} \int_{u=1}^{4} u^2 (e - e^{-1}) \, du.
	\]
	Sacamos el factor constante \( (e - e^{-1}) \):
	\[
	 \frac{e - e^{-1}}{2} \int_{u=1}^{4} u^2 \, du.
	\]
	
	\subsubsection*{Integramos respecto a \( u \) y evaluamos:}
	\[
	\int_{u=1}^{4} u^2 \, du = \left[ \frac{u^3}{3} \right]_{u=1}^{4} = \frac{4^3}{3} - \frac{1^3}{3} = \frac{64}{3} - \frac{1}{3} = \frac{63}{3} = 21.
	\]
	
	\subsection*{Sustituimos el resultado}
	\[
	 \frac{e - e^{-1}}{2} \cdot 21 = \frac{21}{2} (e - e^{-1}).
	\]
	
	\subsection*{Resultado:}
	\[
	\boxed{I = \frac{21}{2} (e - e^{-1})}.
	\]
	
\end{document}
