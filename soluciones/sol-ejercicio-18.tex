\section*{Cálculo del volumen}

Queremos encontrar el volumen de la región encerrada por la superficie \( z = x^2 + y^2 \) y entre \( z = 0 \) y \( z = 10 \).

\subsection*{Paso 1: Conversión a coordenadas cilíndricas}

En coordenadas cilíndricas:
\[
x = r\cos\theta, \quad y = r\sin\theta, \quad z = z,
\]
y la ecuación \( z = x^2 + y^2 \) se convierte en:
\[
z = r^2,
\]
ya que \( x^2 + y^2 = r^2 \).

Los límites son:
\begin{itemize}
    \item \( 0 \leq r \leq \sqrt{10} \) (de \( z = r^2 \) y \( z \leq 10 \)),
    \item \( 0 \leq \theta \leq 2\pi \) (rotación completa alrededor del eje \( z \)),
    \item \( 0 \leq z \leq r^2 \).
\end{itemize}

El elemento de volumen en coordenadas cilíndricas es:
\[
dV = r \, dz \, dr \, d\theta.
\]

\subsection*{Paso 2: Configuración de la integral}

El volumen \( V \) está dado por:
\[
V = \int_0^{2\pi} \int_0^{\sqrt{10}} \int_0^{r^2} r \, dz \, dr \, d\theta.
\]

\subsection*{Paso 3: Evaluación de la integral}

Primero, integramos respecto a \( z \):
\[
\int_0^{r^2} r \, dz = r[z]_0^{r^2} = r(r^2 - 0) = r^3.
\]

Sustituyendo en la integral, tenemos:
\[
V = \int_0^{2\pi} \int_0^{\sqrt{10}} r^3 \, dr \, d\theta.
\]

A continuación, integramos respecto a \( r \):
\[
\int_0^{\sqrt{10}} r^3 \, dr = \left[\frac{r^4}{4}\right]_0^{\sqrt{10}} = \frac{(\sqrt{10})^4}{4} - \frac{0^4}{4} = \frac{100}{4} = 25.
\]

Sustituyendo nuevamente, obtenemos:
\[
V = \int_0^{2\pi} 25 \, d\theta.
\]

Finalmente, integramos respecto a \( \theta \):
\[
\int_0^{2\pi} 25 \, d\theta = 25[\theta]_0^{2\pi} = 25(2\pi - 0) = 50\pi.
\]

\subsection*{Respuesta final}

El volumen de la región es:
\[
\boxed{50\pi}.
\]
