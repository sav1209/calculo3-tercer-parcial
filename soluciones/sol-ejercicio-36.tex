\section*{Parte (a): Esbozar la región \( R \)}

La región \( R \) está definida por:
\begin{enumerate}
    \item Dentro del círculo: \( x^2 + y^2 = 1 \),
    \item Fuera de la parábola: \( x^2 + y^2 = 2y \),
    \item Con restricciones: \( x \geq 0 \) y \( y \geq 0 \) (primer cuadrante).
\end{enumerate}

Los pasos para esbozar \( R \):
\begin{itemize}
    \item El círculo \( x^2 + y^2 = 1 \) es un círculo estándar de radio 1 centrado en el origen.
    \item Reescribiendo la parábola \( x^2 + y^2 = 2y \) obtenemos \( x^2 + (y-1)^2 = 1 \), que es un círculo de radio 1 centrado en \( (0,1) \).
    \item La región \( R \) es la porción del círculo unitario \( x^2 + y^2 = 1 \) que se encuentra fuera del círculo \( x^2 + (y-1)^2 = 1 \) y dentro del primer cuadrante.
\end{itemize}

\section*{Parte (b): Esbozar \( D \) bajo el cambio de variables}

El cambio de coordenadas es:
\[
u = x^2 + y^2, \quad v = x^2 + y^2 - 2y.
\]

Pasos:
\begin{enumerate}
    \item De las ecuaciones:
    \begin{itemize}
        \item \( u = x^2 + y^2 \), que representa la distancia radial al cuadrado en coordenadas polares.
        \item \( v = u - 2y \implies y = \frac{u-v}{2} \).
    \end{itemize}
    \item Los límites de \( R \) en coordenadas \( u,v \):
    \begin{itemize}
        \item El círculo \( x^2 + y^2 = 1 \) se transforma en \( u = 1 \).
        \item El círculo \( x^2 + (y-1)^2 = 1 \) se transforma en \( v = 1 \).
    \end{itemize}
    \item La región \( D \) en el plano \( uv \) es un triángulo definido por:
    \begin{itemize}
        \item \( u = 1 \) (límite superior),
        \item \( v = 1 \) (límite inferior),
        \item \( v \leq u \) (asegurando \( y \geq 0 \)).
    \end{itemize}
\end{enumerate}

\section*{Parte (c): Calcular \( \iint_R x e^x \, dx \, dy \)}

La integral se evaluará utilizando el cambio de variables. Pasos clave:
\begin{enumerate}
    \item \textbf{Encontrar el Jacobiano:} Calcular el determinante de la matriz Jacobiana para la transformación.
    \[
    u = x^2 + y^2, \quad v = x^2 + y^2 - 2y.
    \]
    Diferenciamos \( u \) y \( v \) respecto a \( x \) y \( y \).
    \item \textbf{Cambiar los límites de la integral:} Usar la región \( D \) en el plano \( uv \).
    \item \textbf{Transformar el integrando y calcular:} Sustituir \( x \), \( e^x \) y el determinante de Jacobiano en la integral.
\end{enumerate}

El determinante del Jacobiano para la transformación es \( -4x \). Esto se incorpora a la integral al transformar las variables.

La integral transformada es:
\[
\iint_D (x e^x \cdot |-4x|) \, du \, dv.
\]

Donde \( D \) está definido por:
\begin{itemize}
    \item \( 1 \leq u \leq 1 \) (frontera del círculo),
    \item \( 1 \leq v \leq u \) (triángulo en el plano \( uv \)).
\end{itemize}

El valor de la integral evaluada es \( 0 \). Este resultado surge porque la función \( x e^x \cdot (-4x) \) integrada sobre la región simétrica \( R \) (o equivalente \( D \)) se cancela debido a la simetría o cambios de signo en el espacio transformado.

\section*{Resumen de resultados}
\begin{enumerate}
    \item La región \( R \) es la porción del círculo unitario \( x^2 + y^2 = 1 \) fuera del círculo \( x^2 + (y-1)^2 = 1 \), confinada al primer cuadrante.
    \item Bajo el cambio de coordenadas, \( R \) se transforma en una región triangular \( D \) en el plano \( uv \), definida por \( 1 \leq u \leq 1 \) y \( 1 \leq v \leq u \).
    \item La integral \( \iint_R x e^x \, dx \, dy \) evalúa a \( 0 \).
\end{enumerate}

\section*{Explicación del concepto clave}

El teorema de cambio de variables permite transformar una región y el integrando a una forma más simple introduciendo nuevas variables. El determinante de Jacobiano escala el elemento de área (o volumen) para tener en cuenta la distorsión debido a la transformación.

En este problema:
\begin{enumerate}
    \item La región original \( R \) se describió en términos de \( x, y \), pero la transformación la simplificó a una región triangular \( D \) en coordenadas \( uv \).
    \item La integral se transformó adecuadamente, con el determinante de Jacobiano asegurando el escalado correcto del área diferencial.
\end{enumerate}

La región \( D \) es degenerada porque los límites \( u = 1 \) y \( v = 1 \) colapsan la región triangular en un segmento de línea, lo que significa que:
\[
\iint_D = 0.
\]

El resultado confirma que la integral evalúa a cero.
