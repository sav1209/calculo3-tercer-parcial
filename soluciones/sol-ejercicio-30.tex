\section*{ Solución a)}
	
	La región \( B \) está definida por:
	
	\begin{enumerate}
		\item  \( 0 \leq r \leq 1 \) (por el cilindro \( x^2 + y^2 = 1 \), es decir, la base del cilindro).
		\item  \( 0 \leq \theta \leq 2\pi \) (cubre todo el círculo en el plano XY ).
		\item  \( 0 \leq z \leq r \) (por debajo del cono \( z = \sqrt{x^2 + y^2} \), \( z \) varía desde 0 ()en el plano \( xy \)) hasta \( r \) en la superficie del cono).
	\end{enumerate}
	
	La integral triple en coordenadas cilíndricas es:
	
	\[
	\int_0^{2\pi} \int_0^1 \int_0^r z \, r \, dz \, dr \, d\theta.
	\]
	
	Evaluamos la integral en \( z \):
	
	\[
	\int_0^r z \, dz = \left[ \frac{z^2}{2} \right]_0^r = \frac{r^2}{2}.
	\]
	
	La integral se reduce a:
	
	\[
	\int_0^{2\pi} \int_0^1 \frac{r^2}{2} \cdot r \, dr \, d\theta.
	\]
	
	Simplificando:
	
	\[
	\frac{r^2}{2} \cdot r = \frac{r^3}{2}.
	\]
	
	Reescribimos la integral como:
	
	\[
	\int_0^{2\pi} \int_0^1 \frac{r^3}{2} \, dr \, d\theta.
	\]
	
	Evaluamos la integral en \( r \):
	
	\[
	\int_0^1 \frac{r^3}{2} \, dr = \frac{1}{2} \left[ \frac{r^4}{4} \right]_0^1 = \frac{1}{2} \cdot \frac{1}{4} = \frac{1}{8}.
	\]
	
	Ahora, evaluamos la integral en \( \theta \):
	
	\[
	\int_0^{2\pi} 1 \, d\theta = 2\pi.
	\]
	
	Finalmente, el valor de la integral es:
	
	\[
	\frac{1}{8} \cdot 2\pi = \boxed {\frac{\pi}{4}}.
	\]
	
	\section*{ Solución b)}
	
	Calculamos la integral:
	\[
	\iiint_W \frac{1}{\sqrt{x^2 + y^2 + z^2}} \, dx\,dy\,dz,
	\]
	donde \(W\) está definida por \( \frac{1}{2} \leq z \leq 1 \) y \(x^2 + y^2 + z^2 \leq 1\).
	
	\subsection*{Cambio a coordenadas cilíndricas}
	Coordenadas cilíndricas:
	\[
	x = r\cos\theta, \quad y = r\sin\theta, \quad z = z, \quad dx\,dy\,dz = r \, dr\,d\theta\,dz,
	\]
	y \(\sqrt{x^2 + y^2 + z^2} = \sqrt{r^2 + z^2}\).\\
	
	Reescribimos la integral como:
	\[
	\int_0^{2\pi} \int_{1/2}^1 \int_0^{\sqrt{1-z^2}} \frac{r}{\sqrt{r^2 + z^2}} \, dr\,dz\,d\theta.
	\]
	
	\subsection*{Integral respecto a \(r\)}
	Sea \( u = r^2 + z^2 \), entonces \( du = 2r \, dr \). Los límites son:
	\[
	r = 0 \implies u = z^2, \quad r = \sqrt{1-z^2} \implies u = 1.
	\]
	La integral queda:
	\[
	\int_0^{\sqrt{1-z^2}} \frac{r}{\sqrt{r^2 + z^2}} \, dr = \frac{1}{2} \int_{z^2}^1 u^{-1/2} \, du = \sqrt{1} - \sqrt{z^2} = 1 - z.
	\]
	
	\subsection*{Integral respecto a \(z\)}
	La integral ahora es:
	\[
	\int_{1/2}^1 (1 - z) \, dz = \int_{1/2}^1 1 \, dz - \int_{1/2}^1 z \, dz.
	\]
	Resolviendo:
	\[
	\int_{1/2}^1 1 \, dz = \frac{1}{2}, \quad \int_{1/2}^1 z \, dz = \frac{3}{8}.
	\]
	Por lo tanto:
	\[
	\int_{1/2}^1 (1 - z) \, dz = \frac{1}{2} - \frac{3}{8} = \frac{1}{8}.
	\]
	
	\subsection*{Integral respecto a \(\theta\)}
	Finalmente:
	\[
	\int_0^{2\pi} 1 \, d\theta = 2\pi.
	\]
	
	El resultado es:
	\[
	\iiint_W \frac{1}{\sqrt{x^2 + y^2 + z^2}} \, dx\,dy\,dz = 2\pi \cdot \frac{1}{8} = \boxed{\frac{\pi}{4}}.
	\]
