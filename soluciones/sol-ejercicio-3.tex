\noindent La formula para cambiar a coordenadas polares es la siguiente:
\begin{center}
    $x=rcos(\theta)$     $y=rsen(\theta)$
\end{center}

\noindent Calculamos el Jacobiano:
\begin{center}
    $J = det(\begin{bmatrix}
        \frac{dx}{dr} & \frac{dx}{d\theta} \\
        \frac{dy}{dr} & \frac{dy}{d\theta}
    \end{bmatrix}$)
    \vspace{6pt}

    $J = det(\begin{bmatrix}
        cos(\theta) & -rsen(\theta) \\
        sen(\theta) & rcos(\theta)
    \end{bmatrix}$)
    \vspace{6pt}
    
    $\iff$ 
    
    $J = rcos^2(\theta) + rsen^2(\theta)$
    \vspace{6pt}
    
    $\iff$
    
    $J = r(sen^2(\theta)+cos^2(\theta))$
    \vspace{6pt}
    
    $\iff$ 
    \vspace{6pt}
    
    $J = r$
\end{center}

\noindent Ahora cambiamos los indices de integración, al ser un disco, podemos usar la siguiente formula:
\begin{center}
    $0 \leq \theta \leq 2\pi$
    \vspace{6pt}

    $0 \leq r \leq radio$ $\iff$ $0 \leq r \leq 1$
\end{center}

\noindent Por lo que la integral se vería asi:
    \begin{center}
        $\displaystyle\iint\limits_{D*} rf(rcos\theta,rsen(\theta) \, dr \, d\theta$

        =
        \[\displaystyle\int\limits_0^{2\pi} \!\!\int\limits_0^1 \ re^{r^2cos^2(\theta) + r^2sen^2(\theta)} \, dr \, d\theta\]
        $\iff$
        \[\displaystyle\int\limits_0^{2\pi} \!\!\int\limits_0^1 \ re^{r^2(cos^2(\theta) + sen^2(\theta))} \, dr \, d\theta\]
        $\iff$
        \[\displaystyle\int\limits_0^{2\pi} \!\!\int\limits_0^1 \ re^{r^2} \, dr \, d\theta\]
    \end{center} 

\noindent Ya podemos comenzar a integrar:

    \[\displaystyle\int\limits_0^{2\pi} \!\!\int\limits_0^1 \ re^{r^2} \, dr \, d\theta \]
    \[\displaystyle u=r^2 \text{ , } \frac{du}{2}=rdr\]
    \[\displaystyle  \int\limits_0^{2\pi} \!\!\frac{1}{2}\int\limits_0^1 \ e^u \, du \, d\theta = \int\limits_0^{2\pi} \frac{1}{2}e^{r^2}\bigg|_0^1 d\theta = \int\limits_0^{2\pi} \frac{e}{2}-\frac{1}{2} \,\, d\theta = \frac{1}{2} \int\limits_0^{2\pi} e-1 \,\, d\theta\]

\noindent Que ya es una integral directa:

    \[\displaystyle \frac{1}{2}\int\limits0^{2\pi} e-1 \,\, d\theta \,\, = \,\, \frac{1}{2}[e\theta - \theta]\bigg|_0^{2\pi} \,\, = \,\, \frac{2\pi e - 2\pi}{2} \,\, = \,\, \frac{2\pi(e-1)}{2} \,\, = \,\, \pi(e-1)\]

\noindent El resultado es: $\pi(e-1)$
