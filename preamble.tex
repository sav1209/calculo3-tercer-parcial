\usepackage{fullpage} % Configuracion de margenes a 1 in
\usepackage{parskip}

% Manejo de caracteres UTF-8
\usepackage[utf8]{inputenc} 

% Configuración de idioma y codificación
\usepackage[spanish,mexico]{babel} 
\usepackage[T1]{fontenc}

% Matemáticas
\usepackage{amsmath}
\usepackage{amsfonts}
\usepackage{amssymb}
\usepackage{amsbsy}

\usepackage{amsthm}
\newtheoremstyle{sol}% Nombre del estilo
	{3pt}		% Espacio superior
	{3pt}		% Espacio inferior
	{}			% Fuente del cuerpo
	{}			% Cantidad de indentación
	{\sffamily\bfseries}	% Fuente del encabezado
	{.}			% Puntuación después del encabezado
	{\newline}		% Espacio después del encabezado
	{}			% 
\theoremstyle{sol}
\newtheorem*{solucion}{Solución}

%Configuración de la fuente (Usa xelatex o lualatex para compilar)
%\usepackage{fontspec}
%\setmainfont{XITS}
%\usepackage{unicode-math}
%\setmathfont{XITS Math}

% Gráficos
\usepackage{graphicx}
\usepackage{rotating}
\usepackage{tikz}
\usepackage{pgfplots}   % Para gráficos avanzados, especialmente gráficos 2D.
\pgfplotsset{compat=1.18}  % Especifica la versión compatible de pgfplots para asegurar que no haya errores.

% Listas enumeradas
\usepackage{enumitem}

% Cajas
\usepackage[most]{tcolorbox}

\newtcolorbox{ejercicio}[2][]{
	colback=white,			% Color de fondo
	colframe=black,			% Color del marco
	colbacktitle=black,		% Color de fondo del título
	coltitle=white,			% Color del título
	fonttitle=\large\sffamily\bfseries,	% Fuente del título
	sharp corners=uphill,	% Esquinas redondeadas y en punta
	arc=3mm,				% Arco de las esquinas redondeadas
	title={Ejercicio #2}, % Título automático con numeración
	enhanced,
	attach boxed title to top left={yshift=-5.25mm, yshifttext=-3.5mm, xshift=4mm,},	% Posición de la caja del titulo
	boxed title style={sharp corners=uphill, colframe=black, arc=3mm, boxsep=2mm,}, 	% Configuración de la caja del título
	#1						% Opciones adicionales
}


% Uso de hipervínculos
\usepackage[
	pdftitle={Tercer Examen Parcial},
]{hyperref}
