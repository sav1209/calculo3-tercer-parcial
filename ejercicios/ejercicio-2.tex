\section{Ejercicio 2}
    Suggest a substitution/transformation that will simplify the following integrands, and find their Jacobians.
    \begin{enumerate}
        \item $\iint_R (5x + y)^3 (x + 9y)^4 \, dA$
        \item $\iint_R x \sin(6x + 7y) - 3y \sin(6x + 7y) \, dA$
    \end{enumerate}

    \subsection*{Parte (a)}
\[
\iint_R (5x + y)^3 (x + 9y)^4 \, dA
\]

\subsubsection*{Identificar la Transformación}
Observamos que los términos \(5x + y\) y \(x + 9y\) se pueden simplificar utilizando la transformación:
\[
u = 5x + y, \quad v = x + 9y.
\]

\subsubsection*{Resolver \(x\) y \(y\) en términos de \(u\) y \(v\)}
De las ecuaciones de transformación, resolvemos para \(x\) y \(y\):
\[
x = \frac{u - v}{44}, \quad y = \frac{9v - 5u}{44}.
\]

\subsubsection*{Calcular el Jacobiano}
El Jacobiano \(J\) se define como:
\[
J = \begin{vmatrix}
\frac{\partial x}{\partial u} & \frac{\partial x}{\partial v} \\
\frac{\partial y}{\partial u} & \frac{\partial y}{\partial v}
\end{vmatrix}.
\]

Calculamos las derivadas parciales:
\[
\frac{\partial x}{\partial u} = \frac{1}{44}, \quad \frac{\partial x}{\partial v} = -\frac{1}{44}, \quad \frac{\partial y}{\partial u} = -\frac{5}{44}, \quad \frac{\partial y}{\partial v} = \frac{9}{44}.
\]

Sustituyendo, obtenemos:
\[
J = \begin{vmatrix}
\frac{1}{44} & -\frac{1}{44} \\
-\frac{5}{44} & \frac{9}{44}
\end{vmatrix} = \frac{1}{44} \cdot \frac{9}{44} - \left(-\frac{1}{44} \cdot -\frac{5}{44}\right) = \frac{9}{1936} - \frac{5}{1936} = \frac{4}{1936} = \frac{1}{484}.
\]

Por lo tanto, el Jacobiano es:
\[
\boxed{\frac{1}{484}}
\]

\subsection*{Parte (b)}
\[
\iint_R x \sin(6x + 7y) - 3y \sin(6x + 7y) \, dA
\]

\subsubsection*{Identificar la Transformación}
Observamos que \(6x + 7y\) aparece repetido en el integrando. Usamos la transformación:
\[
u = 6x + 7y, \quad v = x.
\]

\subsubsection*{Resolver \(x\) y \(y\) en términos de \(u\) y \(v\)}
De las ecuaciones de transformación, resolvemos para \(x\) y \(y\):
\[
x = v, \quad y = \frac{u - 6v}{7}.
\]

\subsubsection*{Calcular el Jacobiano}
El Jacobiano \(J\) se define como:
\[
J = \begin{vmatrix}
\frac{\partial x}{\partial u} & \frac{\partial x}{\partial v} \\\\
\frac{\partial y}{\partial u} & \frac{\partial y}{\partial v}
\end{vmatrix}.
\]

Calculamos las derivadas parciales:
\[
\frac{\partial x}{\partial u} = 0, \quad \frac{\partial x}{\partial v} = 1, \quad \frac{\partial y}{\partial u} = \frac{1}{7}, \quad \frac{\partial y}{\partial v} = -\frac{6}{7}.
\]

Sustituyendo, obtenemos:
\[
J = \begin{vmatrix}
0 & 1 \\\\
\frac{1}{7} & -\frac{6}{7}
\end{vmatrix} = (0)(-\frac{6}{7}) - (1)(\frac{1}{7}) = -\frac{1}{7}.
\]

Por lo tanto, el Jacobiano es:
\[
\boxed{-\frac{1}{7}}
\]
