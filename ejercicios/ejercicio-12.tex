Sea \( D^* \) una región \( v \)-simple en el plano \( uv \), delimitada por \( v = g(u) \) y \( v = h(u) \leq g(u) \) para \( a \leq u \leq b \). Sea \( T:\mathbb{R}^2 \to \mathbb{R}^2 \) la transformación dada por \( x = u \) y \( y = \psi(u, v) \), donde \( \psi \) es de clase \( C^{1} \) y \( \partial \psi / \partial v \) nunca es cero. Suponga que \( T(D^*) = D \) es una región \( y \)-simple; demuestra que si \( f: D \to \mathbb{R} \) es continua, entonces

\[
\iint_D f(x, y) \, dx \, dy = \iint_{D^*} f(u, \psi(u, v)) \begin{vmatrix}
    \frac{\partial \psi}{\partial v}
\end{vmatrix} \, du \, dv.
\]
